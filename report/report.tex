\documentclass[slovene,11pt]{article}
\usepackage[utf8]{inputenc}
\usepackage{babel}

\hyphenpenalty=5000
\tolerance=1000

\title{Analaiza algoritma uniclass}
\author{Jure Ham - 63080514}
\maketitle

\begin{document}
\tableofcontents
\pagebreak

\section{Opis programa}
	Program uniclass je celovit paket namenjen testiranju in nadgradji algoritma uniclass. Grafi"cni vmesnik omogo"ca nalaganje testnih primerov, spreminjanje osnovnih nastavitev algoritma ter sproten prikaz delovanja. Ker je algoritem ra"cunsko zahteven, je program implementiran v javi, ki omogo"ca dober kompromis med hitrostjo izvajanja in hitrostjo razvoja.

	\subsection{Branje podatkov}
		Program podpira nalaganje testnih primerov v formatu tab, ki je osnoven format programskega paketa orange. Implementacija je omejena na zvezne in neurejene diskretne vrednosti ter na eno samo meta vrednost, ki postane ime entitete. \\
		Manjkajo"ci podatki so obravnavani kot posebne vrednosti, kar pomeni, da je razdalja med entiteto z vsemi podatki in entiteto brez podatkov najve"cja, razdalja med dvema entitetama brez podatkov pa je najmanj"sa. Razlog za to odlo"citev je zelo enostavna implementacija.
	
	\subsection{Ra"cunanje matrike razdalj}
	\subsection{Simuliranje sil}
	
\section{Vizualizacija podatkov}

\section{Klasifikacija}

\section{Clustering}

\section{Zaklju"cek}
	
\end{document}
